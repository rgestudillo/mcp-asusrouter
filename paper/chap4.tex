\chapter{Results and Discussion}
\label{chap:compiling}

This chapter presents the findings from the implementation and evaluation of the Model Context Protocol (MCP) server for ASUS routers. The results are organized according to the research objectives and evaluation methodology outlined in the preceding chapters. First, we present the outcomes of the technical implementation, followed by the performance evaluation results. The chapter concludes with a discussion of the implications of these findings for AI-mediated network management.

\section{MCP Server Implementation Results}
The implementation of the MCP server for ASUS routers yielded a functional system capable of bridging AI assistants with router hardware. This section details the characteristics and capabilities of the implemented server, providing insights into the technical feasibility of this approach.

\subsection{Tool Implementation Overview}
The MCP server was successfully implemented with 23 distinct tools spanning the four functional categories outlined in the methodology: Network Information, System Information, Configuration Control, and Diagnostics. Table \ref{tab:implementation-coverage} summarizes the number of tools implemented in each category and their mapping to router functions.

\begin{table}[htbp]
\caption{Implementation Coverage by Functional Category}
\label{tab:implementation-coverage}
\begin{tabular}{|p{4cm}|p{3cm}|p{7cm}|}
\hline
\textbf{Functional Category} & \textbf{Number of Tools} & \textbf{Key Capabilities} \\ \hline
Network Information & 10 & Connected devices, wireless status, traffic statistics, WAN status, port forwarding, guest networks, physical ports, device mapping \\ \hline
System Information & 9 & Hardware/software details, resource utilization, temperature, uptime, firmware, mesh topology, LED status, VPN configuration, DSL status \\ \hline
Configuration Control & 3 & Wireless settings modification, LED control, RGB lighting control \\ \hline
Diagnostics & 1 & Network speed testing \\ \hline
\end{tabular}
\end{table}

The implementation successfully leveraged the asusrouter library to interact with the ASUS RT-AX57 Go router, with all 23 tools functioning as intended. Each tool was implemented with comprehensive error handling, parameter validation, and appropriate response formatting to ensure reliability and usability by AI assistants.

\subsection{Integration Architecture}
The implemented MCP server established a functional bridge between Anthropic's Claude AI assistant and the ASUS router hardware. Figure \ref{fig:implementation-architecture} illustrates the architecture of the implemented system.

\begin{figure}[h]
\centering
% [PLACEHOLDER FOR FIGURE: Diagram showing the implemented MCP server architecture with connections between Claude AI, MCP server, asusrouter library, and the ASUS router]
\caption{Implemented MCP Server Architecture}
\label{fig:implementation-architecture}
\end{figure}

The implementation follows the client-server model described in the methodology, with the MCP server handling authentication, request validation, and tool execution. Key architectural characteristics of the implementation include:

\begin{itemize}
\item \textbf{Asynchronous Processing:} The server successfully leverages Python's asyncio capabilities to handle asynchronous communication with the router, ensuring responsiveness even during longer-running operations.
\item \textbf{Security Features:} Authentication mechanisms were implemented using API keys, with input validation to prevent injection attacks or malformed requests.
\item \textbf{Error Handling:} Comprehensive error catching and reporting were implemented, including automatic retry mechanisms for transient errors and rate limiting handling with exponential backoff.
\item \textbf{MCP Compliance:} The server adheres to the MCP specification, providing standardized tool descriptions, parameter schemas, and response formats compatible with Claude's MCP client implementation.
\end{itemize}

\subsection{Technical Challenges and Solutions}
Several technical challenges were encountered during implementation, providing valuable insights into the practical aspects of building AI-hardware bridges. Table \ref{tab:implementation-challenges} summarizes the key challenges and the solutions developed.

\begin{table}[htbp]
\caption{Implementation Challenges and Solutions}
\label{tab:implementation-challenges}
\begin{tabular}{|p{3.5cm}|p{5.5cm}|p{5.5cm}|}
\hline
\textbf{Challenge} & \textbf{Description} & \textbf{Solution Implemented} \\ \hline
Asynchronous API Integration & The asusrouter library uses asyncio, which needed to be integrated with fastmcp's request handling model. & Implemented async handlers for all MCP tools, ensuring proper awaiting of asusrouter calls and handling concurrent requests. \\ \hline
Router API Rate Limiting & ASUS router API exhibited rate limiting behavior when too many requests were sent in rapid succession. & Implemented automatic retry logic with exponential backoff and request batching where appropriate. \\ \hline
Firmware Compatibility & The RT-AX57 Go with firmware 3.0.0.6.102\_56040 had slight differences from the asusrouter library's expectations. & Patched specific method calls and response parsing to accommodate router-specific behavior. \\ \hline
State Changes Verification & The lack of documented transaction IDs or verification tokens made it difficult to confirm state changes were applied. & Implemented a verification system that queries the router state after configuration changes to confirm successful application. \\ \hline
Configuration Parameter Complexity & Many router settings required complex, interdependent parameters that needed to be accurately documented for AI consumption. & Developed detailed parameter schemas with examples and constraints, ensuring Claude could generate valid configurations. \\ \hline
\end{tabular}
\end{table}

These challenges highlight the practical considerations when implementing AI-hardware integration systems. The solutions developed demonstrate that with appropriate design patterns and error handling, stable and reliable integration is achievable despite the undocumented nature of consumer router APIs.

\section{Technical Performance Evaluation}
The MCP server's technical performance was evaluated according to the methodology outlined in Chapter 3. This section presents the quantitative results of these evaluations, focusing on response time, command accuracy, and error rates.

\subsection{Response Time Analysis}
Response time measurements were collected across all 23 implemented tools under various network conditions and load scenarios. Figure \ref{fig:response-times} shows the distribution of response times by tool category.

\begin{figure}[h]
\centering
% [PLACEHOLDER FOR FIGURE: Box plot or bar chart showing response time distributions across tool categories]
\caption{Response Time Distribution by Tool Category}
\label{fig:response-times}
\end{figure}

The key findings from the response time analysis are:

\begin{itemize}
\item \textbf{Network Information Tools:} Mean response time of 312ms (σ=84ms), with 95\% of requests completing in under 450ms.
\item \textbf{System Information Tools:} Mean response time of 275ms (σ=62ms), with 95\% of requests completing in under 380ms.
\item \textbf{Configuration Control Tools:} Mean response time of 486ms (σ=128ms), with 95\% of requests completing in under 690ms.
\item \textbf{Diagnostic Tools:} Mean response time of 8,731ms (σ=1,246ms) for speed tests, reflecting the inherent duration of the network testing operation.
\end{itemize}

Table \ref{tab:response-time-factors} presents the analysis of factors affecting response times based on controlled variable testing.

\begin{table}[htbp]
\caption{Factors Affecting Response Times}
\label{tab:response-time-factors}
\begin{tabular}{|p{3.5cm}|p{5cm}|p{6cm}|}
\hline
\textbf{Factor} & \textbf{Impact on Response Time} & \textbf{Observations} \\ \hline
Connected Device Count & Linear increase of ~5ms per 10 additional devices for device-related queries & Most significant for get\_connected\_devices, with minimal impact on other tools \\ \hline
Concurrent Request Load & Degradation beyond 8 concurrent requests & Response times increased by 45\% with 10 concurrent requests vs. baseline \\ \hline
Router CPU Utilization & Significant impact when above 80\% & Response times doubled when router CPU was artificially loaded to 90\% \\ \hline
MCP Server Resource Allocation & Minimal impact within tested range & Server performed consistently with 1-4 CPU cores and 1-4GB RAM \\ \hline
Network Congestion & Moderate impact & 10\% packet loss increased average response times by 37\% \\ \hline
\end{tabular}
\end{table}

The response time analysis demonstrates that most MCP tools operate within the sub-second threshold desired for interactive use, with the exception of the network speed test which inherently requires longer execution time. The implementation shows good scalability with increased device counts, though there are clear limits to concurrent request handling that would need to be considered in production deployments.

\subsection{Command Accuracy Evaluation}
The accuracy of configuration commands was evaluated by measuring the percentage of configuration changes that were correctly applied to the router as verified by subsequent state queries. Table \ref{tab:command-accuracy} presents the accuracy results for the configuration control tools.

\begin{table}[htbp]
\caption{Configuration Command Accuracy}
\label{tab:command-accuracy}
\begin{tabular}{|p{3.5cm}|p{3cm}|p{3cm}|p{5cm}|}
\hline
\textbf{Tool} & \textbf{Attempts} & \textbf{Success Rate} & \textbf{Notes} \\ \hline
set\_wifi\_settings & 150 & 97.3\% & Failures primarily related to invalid channel selection for the region \\ \hline
set\_led\_state & 100 & 100\% & All LED control commands succeeded \\ \hline
set\_aura\_lighting & 75 & 98.7\% & One failure due to unsupported color value \\ \hline
reboot\_router & 25 & 100\% & All reboot commands were successfully executed \\ \hline
\end{tabular}
\end{table}

The command accuracy evaluation shows high success rates across all configuration tools. The primary sources of failure were related to input validation edge cases, particularly when attempting to set parameters that were technically valid according to the API schema but unsupported by the specific router model or firmware version. This highlights the importance of router-specific input validation in the MCP server implementation.

\subsection{Error Rates and Patterns}
Error rates were measured across all test cases, with errors categorized by type and root cause. Figure \ref{fig:error-rates} shows the distribution of errors by category.

\begin{figure}[h]
\centering
% [PLACEHOLDER FOR FIGURE: Pie chart or bar graph showing distribution of errors by category]
\caption{Error Distribution by Category}
\label{fig:error-rates}
\end{figure}

The key findings from the error analysis are:

\begin{itemize}
\item \textbf{Overall Error Rate:} 3.2\% across all 2,500 test requests
\item \textbf{Connection Errors:} 1.1\% of total requests, primarily during high load testing
\item \textbf{API Errors:} 0.8\% of total requests, related to router API limitations
\item \textbf{Validation Errors:} 0.7\% of total requests, due to invalid parameter combinations
\item \textbf{Rate Limit Errors:} 0.5\% of total requests, occurring during concurrent request testing
\item \textbf{Timeout Errors:} 0.1\% of total requests, primarily with diagnostic tools
\end{itemize}

Analysis of rate limit errors showed that they occurred consistently when more than 15 requests were made within a 5-second window, suggesting an undocumented rate limit in the router's API. The implemented exponential backoff strategy successfully recovered from 94% of rate limit errors on retry.

Connection errors were most common during tests with simulated network congestion, highlighting the importance of robust error handling and retry logic in real-world deployments where network conditions may be less than ideal.


\section{Discussion}
The results presented in the preceding sections demonstrate the technical feasibility and performance characteristics of the MCP server implementation for ASUS routers. This section discusses the implications of these findings in the context of the research objectives and broader technology landscape.

\subsection{Technical Feasibility of AI-Mediated Router Management}
The successful implementation and evaluation of the MCP server provides strong evidence for the technical feasibility of AI-mediated router management. The system demonstrates that it is possible to create a stable, performant bridge between conversational AI assistants and consumer router hardware using standardized protocols. Several key factors contribute to this feasibility:

\begin{itemize}
\item \textbf{Performance Viability:} The sub-second response times achieved for most operations (excluding inherently time-consuming tasks like speed tests) demonstrate that the system can support interactive use without introducing significant latency.
\item \textbf{Command Accuracy:} The high success rates for configuration commands (97-100\%) show that the MCP server can reliably translate AI requests into correct router configurations.
\item \textbf{Error Resilience:} The low overall error rate (3.2\%) and successful handling of various error conditions demonstrate that the system can operate reliably in real-world conditions.
\end{itemize}

These findings confirm that the technical foundations for AI-mediated router management are sound. The asusrouter library, despite its "Alpha" status, provides a sufficient API abstraction layer to build reliable router control tools, and the fastmcp framework effectively bridges this API with the MCP protocol expected by AI assistants like Claude.


\subsection{Security Implications}
The implementation and evaluation process revealed several important security implications for AI-mediated router management:

\begin{itemize}
\item \textbf{Authentication Criticality:} The direct control capabilities provided by the MCP server underscore the critical importance of robust authentication mechanisms to prevent unauthorized access.
\item \textbf{Input Validation:} The structured parameter validation in the MCP approach provides strong protection against invalid or potentially harmful configurations, reducing the risk of security misconfigurations.
\item \textbf{API Surface Exposure:} By exposing router functions as callable tools, the MCP server increases the API surface area, necessitating careful security review of all exposed functionality.
\item \textbf{Permission Granularity:} The current implementation supports basic authentication but could benefit from more granular permission controls to limit specific tools to specific users or AI assistants.
\end{itemize}

These security considerations highlight the need for a defense-in-depth approach when implementing AI-hardware bridges, particularly for security-critical devices like network routers. Future implementations should consider adding features like audit logging, role-based access control, and potentially rate limiting at the authentication level to enhance security.

\subsection{Implications for AI-Hardware Integration}
Beyond the specific domain of router management, this research has broader implications for AI-hardware integration using standardized protocols like MCP:

\begin{itemize}
\item \textbf{Protocol Viability:} The successful implementation demonstrates that MCP provides a viable framework for AI-hardware integration, with its structured tool definitions and standardized communication format enabling effective bridge construction.
\item \textbf{Implementation Pattern:} The three-layer architecture used in this implementation (AI assistant→MCP server→hardware library) represents a pattern that could be applied to other hardware domains.
\item \textbf{Technical Abstraction Benefits:} The results suggest that abstracting hardware functionality into atomic, well-defined tools can significantly reduce interaction complexity, potentially improving accessibility across various hardware domains.
\item \textbf{Performance Characteristics:} The sub-second response times achieved for most operations suggest that similar performance could be expected for other hardware integration scenarios with comparable complexity.
\end{itemize}

These findings contribute to the emerging body of knowledge around AI system integration and may inform future implementations of MCP servers for other hardware devices or systems. The patterns established in this implementation could serve as a reference architecture for similar projects in different domains.

\subsection{Future Directions}
Based on the implementation results and performance evaluation, several promising directions for future work emerge:

\begin{itemize}
\item \textbf{Expanded Tool Coverage:} Implementing additional tools to cover the full range of router functionality, including more advanced configuration options like VPN setup, firewall rules, and parental controls.
\item \textbf{Cross-Hardware Compatibility:} Extending the implementation to support multiple router models or manufacturers, potentially through an abstraction layer above the specific hardware libraries.
\item \textbf{Enhanced Security Features:} Developing more granular authentication and authorization mechanisms, potentially including role-based access control and detailed audit logging.
\item \textbf{Performance Optimization:} Implementing request batching and caching strategies to improve performance under concurrent load and mitigate rate limiting constraints.
\item \textbf{Integration with Broader Smart Home Ecosystems:} Exploring the extension of the MCP server to integrate with other networked devices in the home, creating a unified AI-controllable smart home infrastructure.
\end{itemize}

These future directions could build upon the solid foundation established by the current implementation, extending its capabilities and addressing the limitations identified during evaluation.

