\chapter{Recommendations}
\label{chap:recommendations}

Based on the findings of this study, several recommendations are proposed for future technical implementation, research, and industry applications. These recommendations aim to address the limitations encountered during this research while extending its capabilities and impact.

For technical implementation, expanding the tool coverage to include more advanced router functionality should be prioritized. This expansion should focus on security-critical features such as firewall rule configuration, intrusion detection settings, and VPN setup that could particularly benefit from AI-mediated management. Developing a hardware abstraction layer to support multiple router models and manufacturers would significantly increase the practical utility of the approach. Such an abstraction layer should normalize common functionality across different hardware platforms while preserving access to unique capabilities of specific devices. Additionally, implementing more sophisticated authentication mechanisms including role-based access control and multi-factor authentication for security-sensitive operations would enhance the deployment readiness of MCP servers in production environments.

Performance optimization should focus on developing intelligent request batching and caching strategies to improve response times under concurrent load and mitigate the impact of API rate limiting. Implementing a response prioritization system could ensure that critical network management commands maintain low latency even during periods of high system utilization. Long-running diagnostic operations like speed tests should be designed as asynchronous operations with appropriate client notification mechanisms to improve overall system responsiveness.

Future research directions should include comprehensive user experience studies comparing AI-mediated network management with traditional interfaces to quantify potential improvements in technical accessibility, task completion rates, and configuration accuracy. Longitudinal studies in real-world home and small business environments would provide valuable insights into adoption patterns, learning curves, and long-term effectiveness of AI-mediated network management. Security researchers should conduct thorough vulnerability assessments of the MCP server architecture to identify and address potential security risks introduced by the AI-hardware integration layer.

For router manufacturers and networking equipment providers, integrating standardized AI-interaction capabilities directly into their firmware would eliminate the need for third-party bridge servers and potentially improve both security and performance. Manufacturers should consider developing official, documented APIs for router management to facilitate safer and more reliable integration with external systems. Industry-wide standardization efforts for hardware control protocols compatible with AI systems could accelerate adoption and improve interoperability across different devices and manufacturers.

Educational institutions teaching network management and cybersecurity should incorporate AI-mediated network management approaches into their curricula to prepare students for this emerging paradigm of infrastructure management. Course materials should emphasize both the technical aspects of implementing such systems and the human factors involved in designing effective AI-mediated interfaces for complex technical tasks.

For policymakers and standards organizations, developing guidelines for secure AI integration with critical infrastructure systems like network equipment should be a priority. These guidelines should address authentication requirements, audit logging, access controls, and user privacy considerations. Additionally, accessibility standards could be expanded to recognize AI-mediated interfaces as a potential approach to making complex technical systems more accessible to users with varying levels of technical expertise.

The integration of AI-mediated network management with broader smart home and IoT ecosystems represents a particularly promising direction. Future implementations should explore unified management interfaces that extend beyond router configuration to include connected devices, home automation systems, and other network-connected infrastructure. This integration could enable more sophisticated network optimization based on usage patterns, device requirements, and user preferences.

In conclusion, while this research has demonstrated the technical feasibility of AI-mediated router management through the Model Context Protocol, realizing its full potential will require coordinated efforts across technical implementation, research methodology, industry adoption, education, and policy development. These recommendations provide a roadmap for advancing this approach toward broader practical application and impact.
