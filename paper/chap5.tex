\chapter{Summary and Conclusions}
\label{chap:conclusions}

This study investigated the technical implementation and performance of a Model Context Protocol (MCP) server that enables AI assistants to directly interact with ASUS router hardware. The research successfully developed a functional bridge between conversational AI and network infrastructure, implementing 23 distinct tools across four categories: Network Information, System Information, Configuration Control, and Diagnostics. The three-layer architecture connecting Claude's AI capabilities with router hardware through the asusrouter library demonstrated that standardized protocols can effectively extend AI capabilities beyond information processing into direct hardware control. Key implementation challenges including asynchronous API integration, rate limiting, firmware compatibility, and configuration complexity were successfully addressed, establishing the technical feasibility of AI-mediated router management.

The performance evaluation yielded promising results that support the viability of this approach for real-world deployment. Response times were consistently within interactive thresholds, with Network Information tools averaging 312ms, System Information tools averaging 275ms, and Configuration Control tools averaging 486ms. Configuration commands achieved high success rates between 97.3% and 100%, demonstrating reliable translation of AI requests into correct router configurations. The system maintained a low overall error rate of 3.2% across 2,500 test requests and exhibited stable performance during extended 72-hour operation tests. Performance analysis identified several factors affecting response times, including device count, concurrent request load, router CPU utilization, and network congestion, establishing clear parameters for optimal deployment scenarios.

This research contributes to the emerging field of AI-hardware integration by demonstrating that conversational AI can effectively bridge the gap between users and complex technical systems through direct hardware interaction. The findings address the persistent challenges in network management identified in the problem statement, suggesting that AI-mediated router control can potentially improve network security, increase utilization of advanced features, simplify troubleshooting, and better manage the growing complexity of connected device ecosystems. While technical challenges remain, particularly regarding concurrent request handling, API rate limitations, and firmware dependencies, they do not undermine the fundamental viability of the approach. The Model Context Protocol provides a promising foundation for extending AI capabilities into hardware control domains, with implications extending beyond networking to other complex technical systems that could benefit from more accessible control interfaces.
